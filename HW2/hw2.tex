\documentclass[letterpaper, 11pt]{article}
\usepackage{latexsym}
\usepackage{amssymb}
\usepackage{times}
\usepackage{amsmath,amsfonts,amsthm}

\newcommand*{\defeq}{\mathrel{\vcenter{\baselineskip0.5ex \lineskiplimit0pt
                     \hbox{\scriptsize.}\hbox{\scriptsize.}}}%
                     =}

\begin{document}

\renewcommand{\theenumi}{\alph{enumi}}



\title{600.464 Randomized and Big Data Algorithms \\ Homework \#2 Answers}
\author{Ravindra Gaddipati}


\maketitle

%%%%%%%%%%%%%%%%%%%%%%%%%%%%%%%%%%%%%%%%%%%%%%%%%%%%%%%%%%%%%%%%%%%%%%%%%%%%%%%%%%
\pagebreak
\section*{Problem 1 (3 points)}
Alice and Bob play checkers often. Alice is a better player. so the probability that she wins any given game is 0.6, independent of all other games. They decide to play a tournament of $n$ games. Bound the probability that Alice loses the tournament using a Chernoff bound. \\
\textbf{Answer:} \\
If Alice loses, she wins less than half the games. Let $X$ be the number of games Alice wins, we want to bound $P(X < n/2)$.

%%%%%%%%%%%%%%%%%%%%%%%%%%%%%%%%%%%%%%%%%%%%%%%%%%%%%%%%%%%%%%%%%%%%%%%%%%%%%%%%%%
\pagebreak
\section*{Problem 2 (3 points)}
\begin{enumerate}
	\item In an election with two candidates using paper ballots, each vote is independently misrecorded with probability p = 0.02. Use a Chernoff bound to bound the probability that more than 4\% of the votes are misrecorded in an election of 1,000,000 ballots.
	\item Assume that a misrecorded ballot always counts as a vote for the other candidate. Suppose that candidate A received 5lO,000 votes and that candidate B received 490,000 votes. Use Chernoff bounds to bound the probability that candidate B wins the election owing to misrecorded ballots. Specifically, let $X$ be the number of votes for candidate A that are misrecorded and let $Y$ be the number of votes for candidate B that are misrecorded. Bound $Pr((X > k) \cap (Y < l))$ for suitable choices of $k$ and $e$. 
\end{enumerate}
\textbf{Answer:} \\


%%%%%%%%%%%%%%%%%%%%%%%%%%%%%%%%%%%%%%%%%%%%%%%%%%%%%%%%%%%%%%%%%%%%%%%%%%%%%%%%%%
\pagebreak
\section*{Problem 3 (3 points)}
Recall that a function $f$ is said to be \emph{convex} if for any $x1$, $x2$ and for $0 \leq \lambda \leq 1$,
$$f(\lambda x_1 + (1-\lambda)x_2) \leq \lambda f(x_1) + (1-\lambda)f(x_2)$$
\begin{enumerate}
	\item Let $Z$ be a random variable that takes on a (finite) set of values in the interval [0,1], and let $p=E[Z]$. Define the Bernoulli random variable $X$ by $Pr(X=1)=p$ and $Pr(X=0)=1-p$, show that $E[f(Z)] \leq E[f(X)]$ for any convex function $f$.
	\item Use the fact that $f(x) = e^{tx}$ is convex for any $t \geq 0$ to obtain a Chernoff-like bound for $Z$ baesd on a Chernoff bound for $X$.
\end{enumerate}
\textbf{Answer:} \\


%%%%%%%%%%%%%%%%%%%%%%%%%%%%%%%%%%%%%%%%%%%%%%%%%%%%%%%%%%%%%%%%%%%%%%%%%%%%%%%%%%
\pagebreak
\section*{Problem 4 (3 points)}
Let $X_0=0$ and for $j \geq 0$ let $X_{j+1}$ be chosen uniformly over the real interval $[X_j,1]$. Show that, for $k \geq 0$, the sequence
$$Y_k=2^k(1-X_k)$$
is a martingale. \\
\textbf{Answer:} \\


%%%%%%%%%%%%%%%%%%%%%%%%%%%%%%%%%%%%%%%%%%%%%%%%%%%%%%%%%%%%%%%%%%%%%%%%%%%%%%%%%%
\pagebreak
\section*{Problem 5 (3 points)}
Consider a random graph from $G_{n,N}$, where $N=cn$ for some constant $c>0$. Let $X$ be the expected number of isolated verticies (i.e., verticies of degree 0).
\begin{enumerate}
	\item Determine $E[X]$
	\item Show that:
		$$ P(|X-E[X]| \geq 2\lambda \sqrt{cn}) \leq 2e^{- \lambda^2/2}$$
\end{enumerate}
\textbf{Answer:} \\


\end{document}
